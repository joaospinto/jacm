\section{Generalised Continuous Orbit and Skolem Problems}

We start by recalling the definition of the Generalised Continuous Orbit and Skolem Problems.

\generalisedproblems*

We will now prove that these are both undecidable.

\begin{theorem}
The Generalised Continuous Orbit Problem is undecidable.
\end{theorem}

\begin{proof}
This can be shown by reduction from MESP. In particular, given square matrices $B_{1}, \ldots, B_{k}, C$, we construct matrices $A_{1}, \ldots, A_{k}$ and vectors $\myvector{x}, \myvector{y}$ for which
\begin{equation}
  \label{eq:mesp-to-orbit}
\prod \limits_{i=1}^{k} \exp(B_{i} t_{i}) = C \Leftrightarrow
\prod \limits_{i=1}^{k} \exp(A_{i} t_{i}) \myvector{x} = \myvector{y} .
\end{equation}

Let $\myvector{c}_{1}, \ldots, \myvector{c}_{n}$ be the columns of $C$, from left to right, and let $\myvector{e}_{1}, \ldots, \myvector{e}_{n}$ denote the canonical basis of $\Reals^{n}$. Then~\eqref{eq:mesp-to-orbit} can be achieved by setting, for each $i \in \lbrace 1, \ldots, n \rbrace$,
\begin{equation*}
A_{i} =
\begin{pmatrix}
B_{i} && \cdots && 0 \\
\vdots && \ddots && \vdots \\
0 && \cdots && B_{i}
\end{pmatrix},
\end{equation*}
as well as
\begin{equation*}
\myvector{x} = \begin{pmatrix} \myvector{e}_{1} \\ \vdots \\ \myvector{e}_{n} \end{pmatrix} \mbox{ and }
\myvector{y} = \begin{pmatrix} \myvector{c}_{1} \\ \vdots \\ \myvector{c}_{n} \end{pmatrix} \, .
\end{equation*}
\end{proof}

\begin{theorem}
The Generalised Continuous Skolem Problem is undecidable.
\end{theorem}

\begin{proof}
This can be shown by reduction from the Generalised Continuous Orbit Problem.
Similarly to what we did in the previous proof, we will define matrices $C_{1}, \ldots, C_{k}$ and vectors $\myvector{w}, \myvector{z}$ such that
\begin{equation*}
\prod \limits_{i=1}^{k} \exp(A_{i} t_{i}) \myvector{x} = \myvector{y}
\Leftrightarrow
\myvector{w}^{T} \prod\limits_{i=1}^{k} \exp(C_{i} t_{i}) \myvector{z} = 0 \, .
\end{equation*}

Let $\myvector{e}_{1}, \ldots, \myvector{e}_{n}$ denote the canonical basis of $\Reals^{n}$. Moreover, let
\begin{equation*}
B_{i} = \begin{pmatrix} A_{i} && 0 \\ 0 && 0 \end{pmatrix},
\myvector{u}_{j} = \begin{pmatrix} \myvector{e}_{j} \\ - \myvector{e}_{j} \end{pmatrix},
\myvector{v} = \begin{pmatrix} \myvector{x} \\ \myvector{y} \end{pmatrix}
\end{equation*}
and
\begin{equation*}
C_{i} = \begin{pmatrix} B_{i} \otimes I + I \otimes B_{i} && \cdots && 0 \\ \vdots && \ddots && \vdots \\ 0 && \cdots && B_{i} \otimes I + I \otimes B_{i} \end{pmatrix} \, .
\end{equation*}
Then
\begin{align*}
&\prod \limits_{i=1}^{k} \exp(A_{i} t_{i}) \myvector{x} = \myvector{y} \\
\Leftrightarrow &\sum \limits_{j=1}^{n} \left( \myvector{u}_{j}^{T} \prod \limits_{i=1}^{k} \exp(B_{i} t_{i}) \myvector{v} \right)^{2} = 0 \\
\Leftrightarrow &\sum \limits_{j=1}^{n} \left( \left( \myvector{u}_{j} \otimes \myvector{u}_{j} \right)^{T} \prod \limits_{i=1}^{k} \left( \exp( B_{i} t_{i} ) \otimes \exp(B_{i} t_{i}) \right) \left( \myvector{v} \otimes \myvector{v} \right) \right) = 0 \\
\Leftrightarrow &\sum \limits_{j=1}^{n} \left( \left( \myvector{u}_{j} \otimes \myvector{u}_{j} \right)^{T} \prod \limits_{i=1}^{k} \exp\left((B_{i} \otimes I + I \otimes B_{i}) t_{i} \right) \left( \myvector{v} \otimes \myvector{v} \right) \right) = 0 \\
\Leftrightarrow &\begin{pmatrix} \left(\myvector{u}_{1} \otimes \myvector{u}_{1}\right)^{T} \\ \vdots \\ \left(\myvector{u}_{n} \otimes \myvector{u}_{n}\right)^{T} \end{pmatrix}^{T} \prod \limits_{i=1}^{k} \exp( C_{i} t_{i} ) \begin{pmatrix} \myvector{v} \otimes \myvector{v} \\ \vdots \\ \myvector{v} \otimes \myvector{v} \end{pmatrix} = 0 .
\end{align*}

The result then follows by taking
\begin{equation*}
  \myvector{w} = \begin{pmatrix} \left(\myvector{u}_{1} \otimes \myvector{u}_{1}\right)^{T} \\ \vdots \\ \left(\myvector{u}_{n} \otimes \myvector{u}_{n}\right)^{T} \end{pmatrix}
  \mbox{ and } \myvector{z} = \begin{pmatrix} \myvector{v} \otimes \myvector{v} \\ \vdots \\ \myvector{v} \otimes \myvector{v} \end{pmatrix} \, .
\end{equation*}
\end{proof}
