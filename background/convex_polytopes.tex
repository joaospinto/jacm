\subsubsection{Convex Polytopes}

A \emph{convex polytope} is a subset of $\Reals^{n}$ of the form
\begin{equation*}
\mathcal{P} = \lbrace \myvector{x} \in \Reals^{n} : A \myvector{x} \leq \myvector{b} \rbrace ,
\end{equation*}
where $A$ is a $d \times n$ matrix and $\myvector{b} \in \Reals^{d}$. When all the entries of $A$ and coordinates of $\myvector{b}$ are algebraic numbers, the convex polytope $\mathcal{P}$ is said to have an algebraic description.

The decision version of linear programming with canonically-defined algebraic coefficients is in $\exists\Reals$, as the emptiness of a convex polytope can easily be described by a sentence of the form $\exists x_1 \cdots \exists x_n \sigma(\myvector{x})$.

Finally, we note that even though the decision version of linear
programming with rational coefficients is in~\PTIME{}, allowing
algebraic coefficients makes things more complicated. While it has
been shown that the decision version of linear programming
with canonically-defined algebraic coefficients is solvable in time polynomial
in the size of the problem instance and in the degree of the smallest
number field containing all algebraic numbers in each instance~\cite{AdlerB94}, it
turns out that the degree of that extension can
be exponential in the size of the input\footnote{For example, consider the sequence of field extensions $\Rationals(\sqrt{p_{1}}, \ldots, \sqrt{p_{n}})$, where $p_{i}$ denotes the $i$'th prime.}. In other words, the splitting
field of the characteristic polynomial of a matrix can have a degree
which is exponential in the degree of the characteristic polynomial, which makes it hard to place this problem in~\PTIME{}.

We will also make use of the following result:
\begin{theorem}[Minkowski-Weyl]
  Any polytope $\mathcal{P} \subseteq \mathbb{R}^{d}$ can be written as the sum of two sets $\mathcal{H} \subseteq \mathbb{R}^{d}$ and $\mathcal{C} \subseteq \mathbb{R}^{d}$, where $\mathcal{H}$ is a finitely generated convex hull and $\mathcal{C}$ is a finitely generated cone.
\end{theorem}
