\subsubsection{First-Order Theory of the Reals}

Let $\myvector{x}=(x_1,\ldots,x_m)$ be a vector of $m$ real-valued
variables, and let $\sigma(\myvector{x})$ be a Boolean combination
of atomic predicates of the form $g(\myvector{x})\sim 0$, where each
$g(\myvector{x})$ is a polynomial with integer coefficients in the
variables $\myvector{x}$, and $\sim$ is either $>$ or $=$.
The set of true sentences of the form $\phi=Q_1 x_1 \cdots Q_m x_m
\sigma(\myvector{x})$, where $Q_i$ is either $\exists$ or
$\forall$, is called the \emph{first-order theory of the reals}.

A set $S \subseteq \Reals^{n}$ is said to be \emph{semi-algebraic} if it is a Boolean combination of sets of the form $\lbrace \myvector{x} \in \Reals^{n}: p(\myvector{x}) \geq 0\rbrace$, where $p$ is a polynomial with integer coefficients. Equivalently, the semi-algebraic sets are those definable by the quantifier-free first-order formulas over the structure $(\Reals, <, +, \cdot, 0, 1)$.

\begin{theorem}[Tarski-Seidenberg]
\label{thm:tarski}
The first-order theory of the reals admits a constructive method for quantifier elimination. In particular, it is a decidable theory.~\cite{Tar51}
\end{theorem}

From~\cref{thm:tarski}, it follows that the semi-algebraic sets are precisely the first-order definable sets (that is, the use of quantifiers does not augment the class of semi-algebraic sets).

We also remark that our standard representation of algebraic numbers, described in~\cref{sec:alg_num_rep}, allows us to write them explicitly in the first-order theory of the reals, that is, given $\alpha\in\Algebraics$, there exists a sentence $\sigma(x)$ such that $\sigma(x)$ is true if and only if $x=\alpha$. Thus, we allow their use when defining semi-algebraic sets, for simplicity.

The complexity class $\exists\Reals$ is defined as the set of problems having a polynomial-time many-one reduction to the existential theory of the reals. It was shown in~\cite{Canny88} that $\exists\Reals\subseteq \PSPACE$.

The reader seeking a comprehensive study of the computational complexity of quantifier elimination may refer to~\cite{Renegar} and~\cite{BPR06}. We shall make use of the following result by Basu, Pollack, and Roy
\cite{BasuPR96}, which provides an upper bound on the space complexity of quantifier elimination:

\begin{theorem}
\label{thm:quant-elim}
Given a set $\mathcal{Q}=\lbrace q_1,\ldots,q_s\rbrace$ of $s$   polynomials each of degree at most $D$, in $h+d$ variables, and a first-order formula
\begin{equation*}
  \Phi(\myvector x)=Q y_1 \ldots Q y_h   F(q_1(\myvector x,\myvector y),\ldots,q_s(\myvector x,\myvector y)) \, ,
\end{equation*}
where $Q\in\lbrace \exists,\forall\rbrace$, $F$ is a quantifier-free Boolean combination with atomic elements of the form $q_i(\myvector x,\myvector y)\sim 0$, where $\sim \in \lbrace >, = \rbrace$, there exists a quantifier-free formula
\begin{equation*}
\Psi(\myvector{x}) = \bigwedge_{i=1}^{I} \bigvee_{j=1}^{J_i} q_{ij}(\myvector{x}) \sim 0 \, ,
\end{equation*}
where $I\leq {(sD)}^{O(hd)}$, each $J_{i}\leq {(sD)}^{O(d)}$, the degrees of the   polynomials $q_{ij}$ are bounded by $D^{d}$, and the bit-sizes of the heights of the polynomials in the quantifier-free formula are only   polynomially larger than those of $q_1,\ldots,q_s$.
\end{theorem}

We also make use of the following lemmas:
\begin{lemma}
  If $X\subseteq\Reals^{d}$ is semi-algebraic and non-empty, then
  $X\cap\Algebraics^{d}\neq\emptyset$.
\end{lemma}

\begin{proof}
  We prove this result by strong induction on $d$. Since $X$ is
  semi-algebraic, there exists a quantifier-free sentence in the
  first-order theory of the reals $\sigma$ such that $X=\lbrace
  x\in\Reals^{d}\mid \sigma(x)\rbrace$.

  Suppose that $d>1$. Letting $X_1=\lbrace x_d\in\Reals\mid
  \exists
  x_1,\ldots,x_{d-1}\in\Reals^{d-1},\sigma(x_1,\ldots,x_d)\rbrace$
  and since $X_1\neq\emptyset$ is semi-algebraic, by the induction
  hypothesis, there must be $x_d^*\in\Algebraics\cap X_1$. Moreover, we
  can define $X_2=\lbrace (x_2,\ldots,x_d)\in\Reals^{d-1}\mid
  \sigma(x_1^*,x_2,\ldots,x_n)\rbrace$, which is non-empty and
  semi-algebraic, and again by the induction hypothesis there exists some
  $(x_2^*,\ldots,x_d^*)\in\Algebraics^{d-1}\cap X_2$.

  It remains to prove this statement for $d=1$. In that case, $X$ must
  be a finite union of intervals and points.
  Clearly $\Algebraics$ is dense in any interval, and each of these isolated
  points $x$ corresponds to some constraint $g(x)=0$, which implies
  that $x$ must be algebraic, since $g$ has integer coefficients.
\end{proof}

\begin{lemma}
\label{lem:semi-alg-density}
If $X\subseteq\Reals^{d}$ is semi-algebraic, then $X\cap\Algebraics^{d}$ is dense in $X$.
\end{lemma}

\begin{proof}
  Pick $x\in X$ and $\varepsilon>0$ arbitrarily. Let
  $y\in\Rationals^{d}$ be such that $\| x-y \|<\varepsilon/2$. Since
  $B(y,\varepsilon/2)$ is semi-algebraic, so must be $X\cap
  B(y,\varepsilon/2)$, and so this set must contain an algebraic
  point, since it is nonempty ($x$ is in it), and that point must
  therefore be at distance at most $\varepsilon$ from $x$, by the
  triangle inequality.
\end{proof}

\begin{lemma}
\label{lem:semi-alg-closure}
If $X\subseteq\Reals^{d}$ is semi-algebraic, then so is $\closure{X}$\footnote{$\closure{X}$ denotes the topological closure of $X$.}.
\end{lemma}

\begin{proof}
  Let $\sigma$ be a sentence in the first-order theory of the reals such
  that $X=\lbrace x\in\Reals^{d}\mid \sigma(x)\rbrace$. Whence
\begin{equation*}
  \closure{X}=\lbrace x\in\Reals^{d}\mid
\forall \varepsilon>0,\exists y\in\Reals^{d},\sigma(y)\wedge y\in B(x,\varepsilon) \rbrace .
\end{equation*}
\end{proof}

Finally, we will need the following theorem, shown by MacIntyre and Wilkie~\cite{WM}, concerning the decidability of the extension of the first-order theory of the reals with real exponentiation and bounded $\sin$ and $\cos$ functions.
\begin{theorem}[Wilkie and MacIntyre]
\label{thm:wilkie-macintyre}
  If Schanuel's conjecture is true, then, for each $n \in \Naturals$,
  $FO(\Reals, +, \cdot, <, =, \exp \restriction_{\Reals}, \cos\restriction_{[0, n]}, \sin\restriction_{[0,n]})$ is decidable.
\end{theorem}
