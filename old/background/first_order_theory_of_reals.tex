\subsubsection{First-Order Theory of the Reals}

Let $\myvector{x}=(x_1,\ldots,x_m)$ be a vector of $m$ real-valued
variables, and let $\sigma(\myvector{x})$ be a Boolean combination
of atomic predicates of the form $g(\myvector{x})\sim 0$, where each
$g(\myvector{x})$ is a polynomial with integer coefficients in the
variables $\myvector{x}$, and $\sim$ is either $>$ or $=$.
The set of true sentences of the form $\phi=Q_1 x_1 \cdots Q_m x_m
\sigma(\myvector{x})$, where $Q_i$ is either $\exists$ or
$\forall$, is called the \emph{first-order theory of the reals}.

A set $S \subseteq \Reals^{n}$ is said to be \emph{semi-algebraic} if it is a Boolean combination of sets of the form $\lbrace \myvector{x} \in \Reals^{n}: p(\myvector{x}) \geq 0\rbrace$, where $p$ is a polynomial with integer coefficients. Equivalently, the semi-algebraic sets are those definable by the quantifier-free first-order formulas over the structure $(\Reals, <, +, \cdot, 0, 1)$.

\begin{theorem}[Tarski-Seidenberg]
\label{thm:tarski}
The first-order theory of the reals admits a constructive method for quantifier elimination. In particular, it is a decidable theory.~\cite{Tar51}
\end{theorem}

From~\cref{thm:tarski}, it follows that the semi-algebraic sets are precisely the first-order definable sets (that is, the use of quantifiers does not augment the class of semi-algebraic sets).

We also remark that the standard representation of algebraic numbers\footnote{This consists of their defining polynomial together with a numeric rational approximation of the root and an upper bound on the approximation error that distinguishes it from other roots.} allows us to write them explicitly in the first-order theory of the reals, that is, given $\alpha\in\Algebraics$, there exists a sentence $\sigma(x)$ such that $\sigma(x)$ is true if and only if $x=\alpha$. Thus, we allow their use when defining semi-algebraic sets, for simplicity.

The complexity class $\exists\Reals$ is defined as the set of problems having a polynomial-time many-one reduction to the existential theory of the reals. It was shown in~\cite{Canny88} that $\exists\Reals\subseteq \PSPACE$.

Finally, we will need the following theorem, shown by MacIntyre and Wilkie~\cite{WM}, concerning the decidability of the extension of the first-order theory of the reals with real exponentiation and bounded $\sin$ and $\cos$ functions.
\begin{theorem}[Wilkie and MacIntyre]
\label{thm:wilkie-macintyre}
  If Schanuel's conjecture is true, then, for each $n \in \Naturals$,
  $FO(\Reals, +, \cdot, <, =, \exp \restriction_{\Reals}, \cos\restriction_{[0, n]}, \sin\restriction_{[0,n]})$ is decidable.
\end{theorem}
