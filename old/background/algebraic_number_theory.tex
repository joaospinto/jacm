\subsubsection{Integral solutions of linear equations with algebraic coefficients}
\label{sec:ant}

In this subsection, we introduce a few basic concepts in algebraic number theory, necessary for understanding~\cref{prop:alg_eqn}.

A complex number $\alpha$ is said to be \emph{algebraic} if it is a zero of some non-zero polynomial with integer coefficients. We denote the set of algebraic numbers by $\Algebraics$. It is well known that $\Algebraics$ is a field. A complex number that is not algebraic is said to be \emph{transcendental}.

Given an algebraic number $\alpha$, $\Rationals(\alpha)$ denotes the smallest number field containing $\alpha$. Moreover, if $K=\Rationals(\alpha)$, then $\alpha$ is said to be a \emph{primitive element} of $K$.

A number field is a field extension of $\Rationals$ that, when viewed as a vector space over $\Rationals$, has finite dimension.
It is well known that all number fields have a primitive element.

Consider a non-zero matrix $K\in\Algebraics^{r\times d}$ and vector $\myvector{k} \in \Algebraics^r$.  The following proposition shows how to compute a representation of the set $\lbrace \myvector{x} \in \Integers^{d} : K\myvector{x} = \myvector{k} \rbrace$.

\begin{proposition}
    \label{prop:alg_eqn}
  Let $S = \lbrace \myvector{x}\in\Integers^{d} : K\myvector{x} =
  \myvector{k} \rbrace$. If $S \neq \emptyset$, then there exist
  $\myvector{x}_{0} \in \Integers^{d}$ and $M \in \Integers^{d \times s}$ such that
  $S = \myvector{x}_{0} + \lbrace M \myvector{y} : \myvector{y} \in \Integers^s \rbrace$.
\end{proposition}

\begin{proof}
  Let $\theta$ denote a primitive element of the number field
  generated by the entries of $K$ and $\myvector{k}$. Let the degree of this extension, which equals the degree of $\theta$, be $D$. Then for $\myvector{x} \in \Integers^{d}$ one can write
\begin{align*}
K \myvector{x} = \myvector{k} &\Leftrightarrow \left( \sum \limits_{i=0}^{D-1} N_{i} \theta^{i} \right) \myvector{x} = \sum \limits_{i=0}^{D-1} \myvector{k}_{i} \theta^{i} \\
&\Leftrightarrow N_{i} \myvector{x} = \myvector{k}_{i}, \forall i \in \lbrace 0, \ldots, D-1 \rbrace ,
\end{align*}
for some integer matrices $N_{0}, \ldots, N_{D-1} \in \Integers^{r \times d}$ and integer vectors $\myvector{k}_{0}, \ldots, \myvector{k}_{D-1} \in \Integers^{r}$.
We take $\myvector{x}_{0}$ to be any solution of this system, and select the columns of $M$ to be a minimal set generating
\begin{equation*}
\mathcal{G} = \lbrace \myvector{x} \in \Integers^{d} : \forall i \in \lbrace 0, \ldots, D-1 \rbrace, N_{i} \myvector{x} = \myvector{0} \rbrace \, .
\end{equation*}
Note that, since $\mathcal{G}$ is a subgroup of the finitely generated abelian group $\Integers^{d}$, $\mathcal{G}$ itself must be finitely generated.
\end{proof}
