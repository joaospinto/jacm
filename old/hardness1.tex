\section{Undecidability of the Non-Commutative Case}
\label{sec:lics_encoding}

In this section we show that the Matrix-Exponential Problem is
undecidable in the case of non-commuting matrices.  We show
undecidability for the most fundamental variant of the problem, as
given in \cref{def:MEP}, in which the matrices have real
entries and the variables $t_{i}$ range over the non-negative reals.
Recall that this problem is decidable in the commutative case by the
results of the previous section.

\subsection{Matrix-Exponential Problem with Constraints}

The proof of undecidability in the non-commutative case is by
reduction from Hilbert's Tenth Problem.  The reduction proceeds via
several intermediate problems.  These problems are obtained by
augmenting MEP with various classes of arithmetic constraints on the
real variables that appear in the statement of the problem.

\begin{definition}
  We consider the following three classes of arithmetic constraints
  over real variables $t_1,t_2,\ldots$:
\begin{itemize}
\item $\mathcal{E}_{\pi\Integers}$ comprises constraints of the form
  $t_{i}\in\alpha+\beta\pi\Integers$, where $\alpha$ and $\beta\neq 0$
  are real-valued constants such that $\cos(2\alpha\beta^{-1})$,
  $\beta$ are both algebraic numbers.
\item $\mathcal{E}_{+}$ comprises linear equations of the form
  $\alpha_1 t_1 + \cdots + \alpha_n t_n = \alpha_0 $, for
  $\alpha_0,\ldots,\alpha_n$ real algebraic constants.
\item $\mathcal{E}_{\times}$ comprises equations of the form
  $t_{\ell} = t_{i} t_{j}$.
\end{itemize}
\end{definition}

These constraints will be useful in defining polynominal relations, as well as in enforcing
that certain variables be integer multiples of $\pi$. They will play a crucial role in the
reduction from Hilbert's Tenth Problem.

A class of constraints $\mathcal{E} \subseteq \mathcal{E}_{\pi\Integers} \cup
\mathcal{E}_{+}\cup \mathcal{E}_{\times}$
induces a generalisation of the MEP problem as follows:
\begin{definition}[MEP with Constraints]
  Given a class of constraints
  $\mathcal{E} \subseteq \mathcal{E}_{\pi\Integers} \cup
  \mathcal{E}_{+}\cup \mathcal{E}_{\times}$,
  the problem MEP$(\mathcal{E})$ is as follows.  An instance consists
  of real algebraic matrices $A_1,\ldots,A_k,C$ and a finite set of
  constraints $E\subseteq\mathcal{E}$ on real variables
  $t_1,\ldots,t_k$.  The question is whether there exist non-negative
  real values for $t_1,\ldots,t_k$ such that
  $\prod_{i=1}^{k} e^{A_{i} t_{i}}=C$ and the constraints $E$ are all
  satisfied.
\label{def:contraintMEP}
\end{definition}

Note that in the above definition of MEP$(\mathcal{E})$ the set of
constraints $E$ only utilises real variables $t_1,\ldots,t_k$
appearing in the matrix equation $\prod_{i=1}^{k} e^{A_{i} t_{i}}=C$.
However, without loss of generality, we can allow constraints to
utilise fresh variables $t_{i}$, for $i>k$, since we can always define a
corresponding matrix $A_i=0$ for such variables for then
$e^{A_{i} t_{i}}=I$ has no effect on the matrix product.  In other words,
we can without loss of generality have constraints in $\mathcal{E}$ with existentially
quantified variables.  In particular, we have the following
useful observations:

\begin{itemize}
\item[\textbullet] We can express inequality constraints of the form
  $t_{i}\neq \alpha$ in  $\mathcal{E}_{+}\cup \mathcal{E}_{\times}$ by
using fresh variables
  $t_{j},t_{\ell}$.  Indeed $t_{i} \neq \alpha$ is satisfied whenever there
  exist values of $t_{j}$ and $t_{\ell}$ such that $t_{i}=t_{j}+\alpha$ and
  $t_{j} t_{\ell}=1$.

\item[\textbullet] By using fresh variables,
  $\mathcal{E}_{+}\cup \mathcal{E}_{\times}$ can express polynomial
  constraints of the form $P(t_1,\ldots,t_n)=t$ for $P$ a polynomial
  with integer coefficients.
\end{itemize}

We illustrate the above two observations in an example.
\begin{example}
  Consider the problem, given matrices $A_1,A_2$ and $C$, to decide
  whether there exist $t_1,t_2 \geq 0$ such that
  \[ e^{A_1t_1}e^{A_2t_2}=C \,\mbox{ and }\, t_1^2-1=t_2, t_2\neq 0 \, .\]
  This is equivalent to the following instance of
  MEP$(\mathcal{E}_{+}\cup\mathcal{E}_{\times})$: decide whether there exist $t_1,\ldots,t_5\geq 0$ such that
\[ \prod_{i=1}^5 e^{A_{i} t_{i}}=C \,\mbox{ and }\, t_1t_1=t_3, t_3-1=t_2, t_2t_4=t_5, t_5=1\]
where $A_1,A_2$ and $C$ are as above and $A_3=A_4=A_5=0$.
% where
% \begin{equation*}
% A_1=\begin{bmatrix}1&2\\0&1\end{bmatrix}, A_2=\begin{bmatrix}0&3\\2&1\end{bmatrix},
% A_3=\begin{bmatrix}-1&\sqrt{2}\\7&\tfrac{3}{42}\end{bmatrix},C=\begin{bmatrix}4&3\\2&1\end{bmatrix}.
% \end{equation*}
\end{example}


% The following observations will prove useful in the following:
% \begin{itemize}
% \item[\textbullet] Since $\mathcal{E}_{+}$ contains equations of the
%   form $t_{i}=t_{j}$ and $t_{i}=\alpha$ and so we can use constraints to fix
%   the value of a variable to be a real algebraic constant and we can
%   ensure that several variables have the same value.

% \end{itemize}

We will make heavy use of the following proposition to combine several
instances of the constrained MEP into a single instance by combining
matrices block-wise.
\begin{proposition}\label{prop:remark}
  Given real algebraic matrices $A_1,\ldots,A_k,C$ and
  $A_1',\ldots,A_k',C'$, there exist real algebraic matrices
  $A_1'',\ldots,A_k''$, $C''$ such that for all $t_1,\ldots,t_k$:
  \[\prod_{i=1}^{k} e^{A_{i}''t_{i}}=C''\qquad\Leftrightarrow\qquad\left(\prod_{i=1}^{k} e^{A_{i} t_{i}}=C\right)\wedge\left(\prod_{i=1}^{k} e^{A_i't_{i}}=C'\right).\]
\label{prop:combine}
\end{proposition}

\begin{proof}
For any $i\in \lbrace 1,\ldots,k \rbrace$, define
\[A_i''=\begin{bmatrix}A_i&0\\0&A_i'\end{bmatrix},\qquad
C''=\begin{bmatrix}C&0\\0&C'\end{bmatrix}.\]
The result follows because the matrix exponential can be computed
block-wise (as is clear from its power series definition):
\[\prod_{i=1}^{k} e^{A_i''t_{i}}=\prod_{i=1}^{k}\begin{bmatrix}e^{A_{i} t_{i}}&0\\0&e^{A_i't_{i}}\end{bmatrix}=
\begin{bmatrix}\prod_{i=1}^{k} e^{A_{i} t_{i}}&0\\0&\prod_{i=1}^{k} e^{A_i't_{i}}\end{bmatrix}.\]
\end{proof}

We remark that in the statement of \cref{prop:remark} the
two matrix equations that are combined are over the same set of
variables.  However, we can clearly combine any two matrix equations
for which the common variables appear in the same order in the
respective products.

The core of the reduction is to show that the constraints in
$\mathcal{E}_{\pi\Integers},\mathcal{E}_{+}$ and
$\mathcal{E}_{\times}$ do not make the MEP problem harder.

\begin{proposition}
  MEP$(\mathcal{E}_{\pi\Integers} \cup \mathcal{E}_{+} \cup
  \mathcal{E}_{\times})$ reduces to MEP$(\mathcal{E}_{+} \cup
  \mathcal{E}_{\times})$.
\label{lem:pi}
\end{proposition}
\begin{proof}
  Let $A_1,\ldots,A_k,C$ be real algebraic matrices and
  $E\subseteq \mathcal{E}_{\pi\Integers} \cup \mathcal{E}_{+} \cup
  \mathcal{E}_{\times}$
  a finite set of constraints on real variables $t_1,\ldots,t_k$.  Since $E$ is
  finite it suffices to show how to eliminate a single constraint
  in $\mathcal{E}_{\pi \Integers} \cap E$ from $E$.

  Let $t_{j}\in\alpha+\beta\pi\Integers$ be a constraint in $E$.  By definition
of  $\mathcal{E}_{\pi\Integers}$
we have
  that $\beta \neq 0$ and that both $\beta$ and $\cos(2\alpha\beta^{-1})$
   are real algebraic.  The Pythagorean theorem implies that $\sin(2\alpha\beta^{-1})$ is also real algebraic.  Now define the following extra
  matrices:
\[A'_j=\begin{bmatrix}0&2\beta^{-1}\\-2\beta^{-1}&0\end{bmatrix},
 C'=\begin{bmatrix}\cos(2\alpha\beta^{-1})&
\sin(2\alpha\beta^{-1})\\-\sin(2\alpha\beta^{-1})&\cos(2\alpha\beta^{-1})\end{bmatrix}.\]
Our assumptions ensure that $A_j'$ and $C'$ are both real algebraic.

We now have the following chain of equivalences:
\begin{align*}
e^{A'_j t_{j}}=C'
&\Leftrightarrow\begin{bmatrix}\cos(2t_{j}\beta^{-1})&
\sin(2t_{j}\beta^{-1})\\-\sin(2t_{j}\beta^{-1})&\cos(2t_{j}\beta^{-1})\end{bmatrix}
=C'\\
&\Leftrightarrow\cos(2t_{j}\beta^{-1})=\cos(2\alpha\beta^{-1}) \\
&\qquad\wedge\sin(2t_{j}\beta^{-1})=\sin(2\alpha\beta^{-1}) \\
  &\Leftrightarrow2\beta^{-1}t_{j}\equiv 2\alpha\beta^{-1} \pmod{2\pi} \\
&\Leftrightarrow t_{j}\in\alpha+\beta\pi\Integers.
\end{align*}
Thus the additional matrix equation $e^{A_j't_{j}}=C'$ is equivalent to
the constraint $t_{j}\in\alpha+\beta\pi\Integers$.  Applying
\cref{prop:combine} we can thus eliminate this constraint.
\end{proof}

\begin{proposition}
  MEP$(\mathcal{E}_{+}\cup\mathcal{E}_{\times})$ reduces to
  MEP$(\mathcal{E}_{+})$.
\label{lem:times}
\end{proposition}

\begin{proof}
  Let $A_1,\ldots,A_k,C$ be real algebraic matrices and
  $E \subseteq\mathcal{E}_{+}\cup\mathcal{E}_{\times}$ a finite set of
  constraints on variables $t_1,\ldots,t_k$.  We proceed as above,
  showing how to remove each constraint in
  $\mathcal{E}_{\times}$ from $E$.  In so doing we potentially increase the
  number and the dimension of matrices and add new constraints from $\mathcal{E}_{+}$.

  Let $t_l=t_{i} t_{j}$ be an equation in $E$.  To eliminate this equation
the first step is to introduce fresh
variables $x,x',y,y',z$ and add the constraints
\[ t_{i}=x,\, t_{j}=y,\, t_{\ell} = z,\]
which are all in $\mathcal{E}_{+}$.  We now add a new matrix equation
over the fresh variables $x,x',y,y',z$ that is equivalent to the
constraint $xy=z$.  Since this matrix equation involves a new set of
variables we are free to the set the order of the matrix products,
which is crucial to express the desired constraint.

The key gadget is the following matrix product equation,
  which holds for  any $x,x',y,y',z\geqslant0$:
\begin{equation*}
\begin{bmatrix}1&0&-z\\0&1&0\\0&0&1\end{bmatrix}
\begin{bmatrix}1&0&0\\0&1&-y'\\0&0&1\end{bmatrix}
\begin{bmatrix}1&x&0\\0&1&0\\0&0&1\end{bmatrix}
\begin{bmatrix}1&0&0\\0&1&y\\0&0&1\end{bmatrix}
\begin{bmatrix}1&-x'&0\\0&1&0\\0&0&1\end{bmatrix}=
\begin{bmatrix}1&x-x'&z-xy\\0&1&y-y'\\0&0&1\end{bmatrix}.
\end{equation*}

Notice that each of the matrices on the left-hand side of the above
equation has a single non-zero off-diagonal entry.  Crucially each
matrix of this form can be expressed as an exponential.  Indeed we can
write the above equation as a matrix-exponential product
\[ e^{B_1z} e^{B_2y'} e^{B_3x} e^{B_4y} e^{B_5x'} = \begin{bmatrix}1&x-x'&z-xy\\0&1&y-y'\\0&0&1\end{bmatrix} \]
for matrices
%\[\begin{array}{r@{}cp{.1cm}r@{}cp{.1cm}r@{}c}
%B_1=&\begin{bmatrix}0&0&-1\\0&0&0\\0&0&0\end{bmatrix} &&
%B_3=&\begin{bmatrix}0&1&0\\0&0&0\\0&0&0\end{bmatrix} &&
%B_5=&\begin{bmatrix}0&-1&0\\0&0&0\\0&0&0\end{bmatrix} \\[10pt]
%B_2=&\begin{bmatrix}0&0&0\\0&0&-1\\0&0&0\end{bmatrix} &&
%B_4=&\begin{bmatrix}0&0&0\\0&0&1\\0&0&0\end{bmatrix} \\
%\end{array}\]
\begin{align*}
&B_1=\begin{bmatrix}0&0&-1\\0&0&0\\0&0&0\end{bmatrix} &&
&B_3=\begin{bmatrix}0&1&0\\0&0&0\\0&0&0\end{bmatrix} &&
B_5=\begin{bmatrix}0&-1&0\\0&0&0\\0&0&0\end{bmatrix} \\[10pt]
&B_2=\begin{bmatrix}0&0&0\\0&0&-1\\0&0&0\end{bmatrix} &&
&B_4=\begin{bmatrix}0&0&0\\0&0&1\\0&0&0\end{bmatrix} \\
\end{align*}
Thus the constraint $xy=z$ can be expressed as
\begin{gather}
e^{B_1z} e^{B_2y'} e^{B_3x} e^{B_4y} e^{B_5x'} = I \, .
\label{eq:fresh}
\end{gather}

Again, we can apply \cref{prop:combine} to combine the
equation (\ref{eq:fresh}) with the matrix equation from the original
problem instance and thus encode the constraint $z=xy$.

% as the example below illustrates:
% \[\exp\left(\begin{bmatrix}0&1&0\\0&0&0\\0&0&0\end{bmatrix}x\right)=\begin{bmatrix}1&x&0\\0&1&0\\0&0&1\end{bmatrix}.\]
% Also note that we arranged so that all cancelations happen with nonnegative variables (thus the $x-x'$ instead
% of $x+x'$ for example). Since the order in which the matrices are multiplied is crucial,
% we create new variables $t_{k+1},\ldots,t_{k+4}$ and the following matrices:
% \[\begin{array}{r@{}cp{.1cm}r@{}c}
% A_{l}'=&\begin{bmatrix}0&0&-1\\0&0&0\\0&0&0\end{bmatrix},&&
% A_{k+1}'=&\begin{bmatrix}0&0&0\\0&0&-1\\0&0&0\end{bmatrix},\\
% A_{k+2}'=&\begin{bmatrix}0&1&0\\0&0&0\\0&0&0\end{bmatrix},&&
% A_{k+3}'=&\begin{bmatrix}0&0&0\\0&0&1\\0&0&0\end{bmatrix},\\
% A_{k+4}'=&\begin{bmatrix}0&-1&0\\0&0&0\\0&0&0\end{bmatrix},&&
% C'=&\begin{bmatrix}1&0&0\\0&1&0\\0&0&1\end{bmatrix}.
%   \end{array}\]
%   We also add the following equations to the problem, to relate the new variables to the
%   ones of the equation:
% \begin{equation}\label{eq:mult:eq}
% t_{k+2}=t_{j},\quad t_{k+2}=t_{i},\quad t_{k+3}=t_{j}, \quad t_{k+4}=t_{i}.
% \end{equation}
% It now follows that the following extra product equation becomes:
% \begin{align*}
% &e^{A_l't_l}\prod_{i=1}^{4}e^{A'_{k+i}t_{k+i}}=C'\wedge\eqref{eq:mult:eq}\\
% &\qquad\Leftrightarrow\begin{bmatrix}1&0&t_l-t_{i}t_{j}\\0&1&0\\0&0&1\end{bmatrix}=
% \begin{bmatrix}1&0&0\\0&1&0\\0&0&1\end{bmatrix}\wedge\eqref{eq:mult:eq}\\
% &\qquad\Leftrightarrow t_l=t_{i}t_{j}\wedge\eqref{eq:mult:eq}.
% \end{align*}

% Since variables $t_{k+1},\ldots,t_{k+4}$ were not part of the original
% instance, the reduced problem is equivalent to the original
% one. Indeed, if the reduced problem is satisfiable then the equation
% above shows that the original one is satisfiable. Conversely, if the
% original instance is satisfiable, we can always find an assignement of
% the extra variables to satisfy \eqref{eq:mult:eq} and make the reduced
% problem satisfiable.
\end{proof}

\begin{proposition}
  MEP$(\mathcal{E}_{+})$ reduces to MEP\@.
\label{lem:plus}
\end{proposition}

\begin{proof}
  Let $A_1,\ldots,A_k,C$ be real algebraic matrices and
  $E \subseteq\mathcal{E}_{+}$ a set of constraints.
  We proceed as above, showing how to eliminate each constraint from
  $E$ that lies in $\mathcal{E}_{+}$, while preserving the set of
  solutions of the instance.

Let $\beta+\sum_{i=1}^{k} \alpha_{i} t_{i}=0$ be an equation in $E$.
Recall that $\beta,\alpha_1,\ldots,\alpha_k$ are real algebraic.
Define the extra matrices $A_1',\ldots,A_k'$ and $C'$ as follows:
\[A_i'=
\begin{bmatrix}
  0&\alpha_{i} \\
  0&0
\end{bmatrix},
\qquad C'=\begin{bmatrix}
1&-\beta \\
0&1\end{bmatrix}.
\]
Our assumptions ensure that $A_1',\ldots,A_k'$ and $C'$ are all real algebraic.
Furthermore, the following extra product equation becomes:
\begin{align*}
\prod_{i=1}^{k} e^{A_i't_{i}}=C &\Leftrightarrow
\prod_{i=1}^{k}\begin{bmatrix}
1&\alpha_{i} t_{i} \\
0&1
\end{bmatrix}=\begin{bmatrix}
1&-\beta \\
0&1
\end{bmatrix}\\
&\Leftrightarrow\sum_{i=1}^{k}\alpha_{i} t_{i}=-\beta \, .
\end{align*}
The result follows by applying \cref{prop:combine}.
\end{proof}

Combining~\cref{lem:pi}, \cref{lem:times}, and \cref{lem:plus} we
have:
\begin{proposition}
  MEP$(\mathcal{E}_{\pi\Integers}\cup\mathcal{E}_{+}\cup\mathcal{E}_{\times})$
  reduces to MEP\@.
\end{proposition}

\subsection{Reduction from Hilbert's Tenth Problem}

\begin{theorem}
  MEP is undecidable in the non-commutative case.
\end{theorem}

\begin{proof}
  We have seen in the previous section that the problem
  MEP$(\mathcal{E}_{\pi\Integers}\cup\mathcal{E}_{+}\cup\mathcal{E}_{\times})$
  reduces to MEP without constraints.  Thus it suffices to reduce
  Hilbert's Tenth Problem to
  MEP$(\mathcal{E}_{\pi\Integers}\cup\mathcal{E}_{+}\cup\mathcal{E}_{\times})$.
  In fact the matrix equation will not play a role in the
  target of this reduction, only the additional constraints!

  Let $P$ be a polynomial of total degree $d$ in $k$ variables with
  integer coefficients. From $P$ we build a homogeneous polynomial $Q$, by
  adding a new variable $\lambda$:
  \[Q(\mathbf{x},\lambda)=\lambda^{d}P\left(\frac{x_1}{\lambda},\ldots,
    \frac{x_k}{\lambda}\right).\]
  Note that $Q$ still has integer coefficients. Furthermore, we have the relationship
\[Q(\mathbf{x},1)=P(\mathbf{x}).\]

As we have seen previously, it is easy to encode $Q$ with constraints,
in the sense that we can compute a finite set of constraints
$E_Q \subseteq \mathcal{E}_{+}\cup\mathcal{E}_{\times}$ utilising variables
$t_0,\ldots,t_m,\lambda$
such that $E$ is
satisfied if and only if $t_0=Q(t_1,\ldots,t_k,\lambda)$. Note that
$E_Q$ may need to utilise variables other
than $t_1,\ldots,t_k$ to do that.  Another
finite set of equations
$E_{\pi\Integers}\subseteq\mathcal{E}_{\pi\Integers}$ is used to
encode that $t_1,\ldots,t_k,\lambda\in\pi\Integers$. Finally,
$E_{=}\subseteq\mathcal{E}_{+}\cup\mathcal{E}_{\times}$ is used to
encode $t_{0}=0$ and $1\leqslant \lambda\leqslant4$.  The
latter is done by adding the polynomial equations $\lambda=1+\alpha^2$ and
$\lambda=4-\beta^2$ for some $\alpha$ and $\beta$. Finally we have the
following chain of equivalences:
\begin{align*}
&\exists t_0,\ldots,\lambda\geqslant0\text{ s.t. }E_Q\cup E_{\pi\Integers}\cup E_{=}\text{ is satisfied }\\
&\qquad\Leftrightarrow\exists t_1,\ldots,\lambda\geqslant0\text{ s.t. } 0=Q(t_1,\ldots,t_k,\lambda)\\
    &\qquad\qquad\wedge t_1,\ldots,t_k,\lambda\in\pi\Integers\wedge 1\leqslant\lambda\leqslant4\\
&\qquad\Leftrightarrow\exists n_1,\ldots,n_k\in\Naturals\text{ s.t. } 0=Q(\pi n_1,\ldots,\pi n_k,\pi)\\
&\qquad\Leftrightarrow\exists n_1,\ldots,n_k\in\Naturals\text{ s.t. } 0=\pi^{d} Q(n_1,\ldots,n_k,1)\\
&\qquad\Leftrightarrow\exists n_1,\ldots,n_k\in\Naturals\text{ s.t. } 0=P(n_1,\ldots,n_k).\\
\end{align*}
\end{proof}
